\documentclass[11pt]{article}
\usepackage{geometry}                % See geometry.pdf to learn the layout options. There are lots.
\geometry{letterpaper}                   % ... or a4paper or a5paper or ... 
%\geometry{landscape}                % Activate for for rotated page geometry
\usepackage[parfill]{parskip}    % Activate to begin paragraphs with an empty line rather than an indent
\setlength\textheight{9.0in}
\setlength\textwidth{6.5in}
\usepackage{graphicx}
\usepackage{amssymb}

\usepackage{epstopdf}
\usepackage{amsmath, amsthm, amssymb}
\usepackage{float}
\newcommand{\vectornorm}[1]{\left|\left|#1\right|\right|}
\newcommand{\reg}[1]{^{(#1)}}
\DeclareGraphicsRule{.tif}{png}{.png}{`convert #1 `dirname #1`/`basename #1 .tif`.png}

\title{MATH 609 Homework \#1}
\author{Hayes Stripling}
%\date{\today}                                           % Activate to display a given date or no date

\begin{document}

\section*{An Analytic Toy Problem for Computing Global Error}
Here's a very simple DAE
\begin{align*}
F(\dot{x},x,p,t)=\left\{\begin{array}{c}y'+p\cdot a(t)y(t)\\t/200-a(t)\end{array}\right\}=0
\end{align*}
The analytic solution for $y$ is
\begin{align*}
Y(t)=\frac{y_0}{\text{exp}\left(\frac{pt^2}{200}\right)}
\end{align*}
Now say our metric is $g(p)=y(t_f)$.  Then 
\begin{align*}
\frac{dg}{dp}=\frac{dy(t_f)}{dp}=\frac{-y_0t_f^2}{200\text{exp}\left(\frac{pt^2}{200}\right)}
\end{align*}
The global error is
\begin{align*}
e(t)=Y(t)-y(t).
\end{align*}
The adjoint equation is
\begin{align*}
\left(\begin{array}{c}\lambda^d\\0\end{array}\right)'&=\left(\begin{array}{cc}(F_{x^d}^d)^T & (F_{x^d}^a)^T\\(F_{x^a}^d)^T & (F_{x^a}^a)^T\end{array}\right)\left(\begin{array}{c}\lambda^T\\\lambda^a\end{array}\right)\\
&=\left(\begin{array}{cc}p\cdot a(t) & 0\\p\cdot y & -1\end{array}\right)\left(\begin{array}{c}\lambda^T\\\lambda^a\end{array}\right)
\end{align*}

\subsection*{Running the matlab code}
You will need the subdirectory \emph{builtIns} as it contains the hacked ODE solvers that I created for extracting truncation eror.  The driver code file is
\begin{align*}
\text{[forSol, backSol]=testAnalyticDAE(doPlot)}
\end{align*}
where doPlot==1 will produce a couple of plots and doPlot==0 will suppress the plots.  The code has some comments which should help you see what I am doing.  As far as how I am extracting/estimating the truncation error, I think I'd rather explain and show in person.

Here is an example code output:\\

True Global Error: -1.4057e-05\\
Est. Global Error: 3.9762e-07 (Rel Error = -1.028e+00)\\
True dgdp: -1.0394e-01\\
Est. dpgp: -1.0394e-01 (Rel Error = 1.683e-05)\\

Feel free to send me questions.
\end{document}

