\section{Airfoil}
\subsection{Author and source}
The airfoil test-case is derived from the paper \cite{Giles_2005}. The source code is available \href{http://people.maths.ox.ac.uk/gilesm/codes/airfoil/bangalore05.tar}{here}. The application is a 2D inviscid airfoil code using an unstructured grid. 
\subsection{Description of the mathematical formulation}
For the mathematical formulation refer \cite{Giles_2005}.

\subsection{Directory structure and description of files}
The airfoil test-case is organized into five subdirectories as shown below:\\
\dirtree{%
.1 /. 
.2 air\_foil\_tapenade\DTcomment{Tapenade in Tangent and Adjoint mode}. 
.2 air\_foil\_wopenad\_joint\DTcomment{OpenAD/F in Reverse Joint mode}. 
.2 air\_foil\_wopenad\_split\DTcomment{OpenAD/F in Reverse Split mode}. 
.2 air\_foil\_wopenad\_tanglin\DTcomment{OpenAD/F in Forward mode}. 
.2 mesh\_generator\DTcomment{MATLAB scripts to generate mesh}. 
}
\subsubsection{Differentiated code using Tapenade}
The directory \texttt{air\_foil\_tapenade} contains the files to compile the original \textbf{non-linear flow code}, the \textbf{forward-mode linear flow code} and the \textbf{reverse-mode adjoint flow code}. The binaries (\textbf{files}), corresponding to these, on building the directory are \texttt{airfoil} (\textbf{{airfoil.F}}), \texttt{air\_lin} (\textbf{{air\_lin.F}}) and \texttt{air\_adj} (\textbf{{air\_adj.F}}) respectively. \\

\noindent The binary \texttt{testlinadj} (\textbf{{testlinadj.F}}), which is built along with the rest, constructs a small mesh, initializes flow and tests the linear and adjoint versions of the routines against one another and each of them against the estimate from performing complex finite difference. \\

\noindent The differentiated versions of the original non-linear flow routines are used in the files \texttt{air\_lin.F} and \texttt{air\_adj.F}. These have been obtained by passing the undifferentiated routines to Tapenade on the \href{http://www-tapenade.inria.fr:8080/tapenade/index.jsp}{web}. \\

\noindent The arguments that were used while generating the differentiated version of the routines can be traced back to the Tapenade invocations that are commented in the \texttt{Makefile}, as Tapenade was not available locally at the time. An example of how one would call tapenade to generate the adjoint version \texttt{flux\_face\_bx.f} appears on the following page. 
\clearpage
\begin{lstlisting}[style=BashInputStyle]
$ tapenade -backward                                   \
		   -head         flux_face                         \
		   -output       flux_face                         \
		   -vars         "x1 x2 q1 q2 adt1 adt2 res1 res2" \
		   -outvars      "x1 x2 q1 q2 adt1 adt2 res1 res2" \
		   -difffuncname "_bx"                             \
		   routines.f;
\end{lstlisting}
\noindent The listing of files and their descriptions follow.\\
\dirtree{%
.1 /. 
.2 air\_foil\_tapenade.  
.3 adBuffer.c\DTcomment{Tapenade support file}. 
.3 adBuffer.f\DTcomment{Tapenade support file}. 
.3 adBuffer.h\DTcomment{Tapenade support file}. 
.3 adStack.c\DTcomment{Tapenade support file}. 
.3 adStack.h\DTcomment{Tapenade support file}. 
.3 air\_adj.F\DTcomment{Driver for reverse-mode adjoint flow code}. 
.3 airfoil.F\DTcomment{Driver for non-linear flow code}. 
.3 air\_lin.F\DTcomment{Driver for forward-mode linear flow code}. 
.3 flux\_face\_b.f\DTcomment{ADJ version of flux\_face from routines.F}. 
.3 flux\_face\_bx.f\DTcomment{Ditto flux\_face\_b with co-ordinate variations}. 
.3 flux\_face\_d.f\DTcomment{TL version of flux\_face from routines.F}. 
.3 flux\_face\_dx.f\DTcomment{Ditto flux\_face\_d with co-ordinate variations}. 
.3 flux\_wall\_b.f\DTcomment{ADJ version of flux\_wall from routines.F}. 
.3 flux\_wall\_bx.f\DTcomment{Ditto flux\_wall\_b with co-ordinate variations}. 
.3 flux\_wall\_d.f\DTcomment{TL version of flux\_wall from routines.F}. 
.3 flux\_wall\_dx.f\DTcomment{Ditto flux\_wall\_d with co-ordinate variations}. 
.3 input.F\DTcomment{Routines to read and write data}. 
.3 lift\_wall\_b.f\DTcomment{ADJ version of lift\_wall from routines.F}. 
.3 lift\_wall\_bx.f\DTcomment{Ditto lift\_wall\_b with co-ordinate variations}. 
.3 lift\_wall\_d.f\DTcomment{TL version of lift\_wall from routines.F}. 
.3 lift\_wall\_dx.f\DTcomment{Ditto lift\_wall\_d with co-ordinate variations}. 
.3 print\_active.F\DTcomment{Routines to record matrices and vectors}. 
.3 routines.F\DTcomment{Real and complex versions of non-linear routines}. 
.3 testlinadj.F\DTcomment{Derivatives test code}. 
.3 time\_cell\_b.f\DTcomment{ADJ version of time\_cell from routines.F}. 
.3 time\_cell\_bx.f\DTcomment{Ditto time\_cell\_b with co-ordinate variations}. 
.3 time\_cell\_d.f\DTcomment{TL version of time\_cell from routines.F}. 
.3 time\_cell\_dx.f\DTcomment{Ditto time\_cell\_d with co-ordinate variations}. 
.3 const.inc\DTcomment{Constants common block}. 
.3 flow.dat\DTcomment{Initial flow - scaled up problem size}. 
.3 flow.dat.bak\DTcomment{Initial flow - original}. 
.3 grid.dat\DTcomment{Unstructured grid - scaled up problem size}. 
.3 grid.dat.bak\DTcomment{Unstructured grid - original}. 
.3 Makefile\DTcomment{Build commands}. 
}
\clearpage
\subsubsection{Differentiated code using OpenAD in Reverse Joint Mode}
For details on reverse joint mode refer \cite{Griewank_2008} and \cite{Utke_2014}.\\

\noindent The directory \texttt{air\_foil\_wopenad\_joint} contains the files to compile the original \textbf{non-linear flow code} and the \textbf{reverse-mode adjoint flow code}. The binaries (\textbf{files}), corresponding to these, on building the directory are \texttt{airfoil} (\textbf{{airfoil.F}}) and \texttt{air\_adj} (\textbf{{air\_adj.F}}) respectively. \\

\begin{TodoPar}\noindent The binary \texttt{testlinadj} (\textbf{{testlinadj.F}}), which was earlier used to validate the forward-linear and reverse-adjoint routines in the \texttt{air\_foil\_tapenade} subdirectory, is not yet available to test OpenAD/F as modifications have to be done to use \texttt{oad\_active} type in the source code.\end{TodoPar}

\noindent The adjoint version of the original non-linear flow routines are used in the file  \texttt{air\_adj.F}. These have been obtained by passing the undifferentiated routines to OpenAD/F in reverse-joint mode.\\

\noindent For details on how to call OpenAD/F in reverse-joint mode refer \cite{Utke_2014}. The listing of files and their descriptions follow.\\

\dirtree{%
.1 /. 
.2 air\_foil\_wopenad\_joint.  
.3 adStack.c.bak\DTcomment{Unused Tapenade support file}. 
.3 air\_adj.F\DTcomment{Driver for reverse-mode adjoint flow code}. 
.3 airfoil.F\DTcomment{Driver for non-linear flow code}. 
.3 air\_lin.F.bak\DTcomment{Unused forward-mode linear flow code driver}. 
.3 const.inc\DTcomment{Constants common block}. 
.3 flow.dat\DTcomment{Initial flow - scaled up problem size}. 
.3 flow.dat.bak\DTcomment{Initial flow - original}. 
.3 flux\_face.F\DTcomment{UD version of flux\_face from routines.F}. 
.3 flux\_wall.F\DTcomment{UD version of flux\_wall from routines.F}. 
.3 grid.dat\DTcomment{Unstructured grid - scaled up problem size}. 
.3 grid.dat.bak\DTcomment{Unstructured grid - original}. 
.3 iaddr.c\DTcomment{OpenAD/F support file}. 
.3 input.F\DTcomment{Routines to read and write data}. 
.3 lift\_wall.F\DTcomment{UD version of lift\_wall from routines.F}. 
.3 Makefile\DTcomment{Build commands}. 
.3 Makefile.bak\DTcomment{Unused makefile}. 
.3 print\_active.F\DTcomment{Routines to record matrices and vectors}. 
.3 routines.F\DTcomment{Real and complex versions of non-linear routines}. 
.3 testlinadj.F.bak\DTcomment{Unused derivatives test code}. 
.3 time\_cell.F\DTcomment{UD version of time\_cell from routines.F}.  
}

\clearpage
\subsubsection{Differentiated code using OpenAD in Reverse Split Mode}
For details on reverse split mode refer \cite{Griewank_2008} and \cite{Utke_2014}.\\

\noindent The directory \texttt{air\_foil\_wopenad\_split} contains the files to compile the original \textbf{non-linear flow code} and the \textbf{reverse-mode adjoint flow code}. The binaries (\textbf{files}), corresponding to these, on building the directory are \texttt{airfoil} (\textbf{{airfoil.F}}) and \texttt{air\_adj} (\textbf{{air\_adj.F}}) respectively. \\

\begin{TodoPar}\noindent The binary \texttt{testlinadj} (\textbf{{testlinadj.F}}), which was earlier used to validate the forward-linear and reverse-adjoint routines in the \texttt{air\_foil\_tapenade} subdirectory, is not yet available to test OpenAD/F as modifications have to be done to use \texttt{oad\_active} type in the source code.\end{TodoPar}

\noindent The adjoint version of the original non-linear flow routines are used in the file  \texttt{air\_adj.F}. These have been obtained by passing the undifferentiated routines to OpenAD/F in reverse-split mode.\\

\noindent For details on how to call OpenAD/F in reverse-split mode refer \cite{Utke_2014}. The listing of files and their descriptions follow.\\

\dirtree{%
.1 /. 
.2 air\_foil\_wopenad\_split.  
.3 adStack.c.bak\DTcomment{Unused Tapenade support file}. 
.3 air\_adj.F\DTcomment{Driver for reverse-mode adjoint flow code}. 
.3 airfoil.F\DTcomment{Driver for non-linear flow code}. 
.3 air\_lin.F.bak\DTcomment{Unused forward-mode linear flow code driver}. 
.3 const.inc\DTcomment{Constants common block}. 
.3 flow.dat\DTcomment{Initial flow - scaled up problem size}. 
.3 flow.dat.bak\DTcomment{Initial flow - original}. 
.3 flux\_face.F\DTcomment{UD version of flux\_face from routines.F}. 
.3 flux\_wall.F\DTcomment{UD version of flux\_wall from routines.F}. 
.3 grid.dat\DTcomment{Unstructured grid - scaled up problem size}. 
.3 grid.dat.bak\DTcomment{Unstructured grid - original}. 
.3 iaddr.c\DTcomment{OpenAD/F support file}. 
.3 input.F\DTcomment{Routines to read and write data}. 
.3 lift\_wall.F\DTcomment{UD version of lift\_wall from routines.F}. 
.3 Makefile\DTcomment{Build commands}. 
.3 Makefile.bak\DTcomment{Unused makefile}. 
.3 print\_active.F\DTcomment{Routines to record matrices and vectors}. 
.3 routines.F\DTcomment{Real and complex versions of non-linear routines}. 
.3 testlinadj.F.bak\DTcomment{Unused derivatives test code}. 
.3 time\_cell.F\DTcomment{UD version of time\_cell from routines.F}.  
}

\subsection{Modifications performed}
\subsection{How to build}
Running make as below, in each of the four subdirectories beginning with ``airfoil'' will build the  binaries \texttt{airfoil}, \texttt{air\_lin} and \texttt{air\_adj}.
\begin{lstlisting}[style=BashInputStyle]
    $ make
\end{lstlisting}
\subsection{How to verify}
\subsection{How to extend}
\subsubsection{Generating larger grids and flows}
\subsection{Todo}