\section{Airfoil}
\subsection{Author and source}
The airfoil test-case is derived from the paper \cite{Giles_2005}. The source code is available \href{http://people.maths.ox.ac.uk/gilesm/codes/airfoil/bangalore05.tar}{here}. The application is a 2D inviscid airfoil code using an unstructured grid. 
\subsection{Description of the mathematical formulation}
For the mathematical formulation refer \cite{Giles_2005}.

\subsection{Directory structure and description of files}
The airfoil test-case is organized into four subdirectories as shown below:\\
\dirtree{%
.1 /. 
.2 air\_foil\_tapenade\DTcomment{Tapenade in Tangent and Adjoint mode}. 
.2 air\_foil\_wopenad\_joint\DTcomment{OpenAD/F in Reverse Joint mode}. 
.2 air\_foil\_wopenad\_split\DTcomment{OpenAD/F in Reverse Split mode}. 
.2 air\_foil\_wopenad\_tanglin\DTcomment{OpenAD/F in Forward mode}. 
}
\subsubsection{Differentiated code using Tapenade}
The directory \texttt{air\_foil\_tapenade} contains the files to compile the original \textbf{non-linear flow code}, the \textbf{forward-mode linear flow code} and the \textbf{reverse-mode adjoint flow code}. The binaries (\textbf{files}) corresponding to these on building the directory are \texttt{airfoil} (\textbf{{airfoil.F}}), \texttt{air\_lin} (\textbf{{air\_lin.F}}) and \texttt{air\_adj} (\textbf{{air\_adj.F}}) respectively. \\

\noindent The differentiated versions of the original non-linear flow routines are used in the files \texttt{air\_lin.F} and \texttt{air\_adj.F}. These have been obtained by passing the undifferentiated routines to Tapenade on the \href{http://www-tapenade.inria.fr:8080/tapenade/index.jsp}{web}. \\

\noindent The listing of files and their descriptions follow.\\
\dirtree{%
.1 /. 
.2 air\_foil\_tapenade.  
.3 adBuffer.c\DTcomment{Tapenade support file}. 
.3 adBuffer.f\DTcomment{Tapenade support file}. 
.3 adBuffer.h\DTcomment{Tapenade support file}. 
.3 adStack.c\DTcomment{Tapenade support file}. 
.3 adStack.h\DTcomment{Tapenade support file}. 
.3 air\_adj.F\DTcomment{Driver for reverse-mode adjoint flow code}. 
.3 airfoil.F\DTcomment{Driver for non-linear flow code}. 
.3 air\_lin.F\DTcomment{Driver for forward-mode linear flow code}. 
.3 flux\_face\_b.f\DTcomment{ADJ version of flux\_face from routines.F}. 
.3 flux\_face\_bx.f\DTcomment{Ditto flux\_face\_b with co-ordinate variations}. 
.3 flux\_face\_d.f\DTcomment{TL version of flux\_face from routines.F}. 
.3 flux\_face\_dx.f\DTcomment{Ditto flux\_face\_d with co-ordinate variations}. 
.3 flux\_wall\_b.f\DTcomment{ADJ version of flux\_wall from routines.F}. 
.3 flux\_wall\_bx.f\DTcomment{Ditto flux\_wall\_b with co-ordinate variations}. 
.3 flux\_wall\_d.f\DTcomment{TL version of flux\_wall from routines.F}. 
.3 flux\_wall\_dx.f\DTcomment{Ditto flux\_wall\_d with co-ordinate variations}. 
.3 input.F\DTcomment{Routines to read and write data}. 
.3 lift\_wall\_b.f\DTcomment{ADJ version of lift\_wall from routines.F}. 
.3 lift\_wall\_bx.f\DTcomment{Ditto lift\_wall\_b with co-ordinate variations}. 
.3 lift\_wall\_d.f\DTcomment{TL version of lift\_wall from routines.F}. 
.3 lift\_wall\_dx.f\DTcomment{Ditto lift\_wall\_d with co-ordinate variations}. 
.3 print\_active.F\DTcomment{Routines to record matrices and vectors}. 
.3 routines.F\DTcomment{Real and complex versions of non-linear routines}. 
.3 testlinadj.F\DTcomment{Validate Tapenade generated routines}. 
.3 time\_cell\_b.f\DTcomment{ADJ version of time\_cell from routines.F}. 
.3 time\_cell\_bx.f\DTcomment{Ditto time\_cell\_b with co-ordinate variations}. 
.3 time\_cell\_d.f\DTcomment{TL version of time\_cell from routines.F}. 
.3 time\_cell\_dx.f\DTcomment{Ditto time\_cell\_d with co-ordinate variations}.  
}

\subsection{Modifications performed}
\subsection{How to build}
Running make as below, in each of the four subdirectories listed above will build the  binaries \texttt{airfoil}, \texttt{air\_lin} and \texttt{air\_adj}.
\begin{lstlisting}[style=BashInputStyle]
    $ make
\end{lstlisting}
\subsection{How to verify}
\subsection{How to extend}
\subsubsection{Generating larger grids and flows}
\subsection{Todo}