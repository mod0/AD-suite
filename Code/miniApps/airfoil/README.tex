\section{Airfoil}
\subsection{Author and source}
The airfoil test-case is derived from the paper \cite{Giles_2005}. The source code is available \href{http://people.maths.ox.ac.uk/gilesm/codes/airfoil/bangalore05.tar}{here}. The application is a 2D inviscid airfoil code using an unstructured grid. 
\subsection{Description of the mathematical formulation}
For the mathematical formulation refer \cite{Giles_2005}.

\subsection{Directory structure and description of files}
The airfoil test-case is organized into five subdirectories as shown below:\\
\dirtree{%
.1 /. 
.2 air\_foil\_tapenade\DTcomment{Tapenade in Tangent and Adjoint mode}. 
.2 air\_foil\_wopenad\_joint\DTcomment{OpenAD/F in Reverse Joint mode}. 
.2 air\_foil\_wopenad\_split\DTcomment{OpenAD/F in Reverse Split mode}. 
.2 air\_foil\_wopenad\_tanglin\DTcomment{OpenAD/F in Forward mode}. 
.2 mesh\_generator\DTcomment{MATLAB scripts to generate mesh}. 
}
\subsubsection{Differentiated code using Tapenade}
The directory \texttt{air\_foil\_tapenade} contains the files to compile the original \textbf{non-linear flow code}, the \textbf{forward-mode linear flow code} and the \textbf{reverse-mode adjoint flow code}. The binaries (\textbf{files}), corresponding to these, on building the directory are \texttt{airfoil} (\textbf{{airfoil.F}}), \texttt{air\_lin} (\textbf{{air\_lin.F}}) and \texttt{air\_adj} (\textbf{{air\_adj.F}}) respectively. \\

\noindent The binary \texttt{testlinadj} (\textbf{{testlinadj.F}}), which is built along with the rest, constructs a small mesh, initializes flow and tests the linear and adjoint versions of the routines against one another and each of them against the estimate from performing complex finite difference. \\

\noindent The differentiated versions of the original non-linear flow routines are used in the files \texttt{air\_lin.F} and \texttt{air\_adj.F}. These have been obtained by passing the undifferentiated routines to Tapenade on the \href{http://www-tapenade.inria.fr:8080/tapenade/index.jsp}{web}. \\

\noindent The arguments that were used while generating the differentiated version of the routines can be traced back to the Tapenade invocations that are commented in the \texttt{Makefile}, as Tapenade was not available locally at the time. An example of how one would call tapenade to generate the adjoint version \texttt{flux\_face\_bx.f} appears on the following page. 
\clearpage
\begin{lstlisting}[language=mybash]
$ tapenade -backward                                   \
		   -head         flux_face                         \
		   -output       flux_face                         \
		   -vars         "x1 x2 q1 q2 adt1 adt2 res1 res2" \
		   -outvars      "x1 x2 q1 q2 adt1 adt2 res1 res2" \
		   -difffuncname "_bx"                             \
		   routines.f;
\end{lstlisting}
\noindent The listing of files and their descriptions follow.\\
\dirtree{%
.1 /. 
.2 air\_foil\_tapenade.  
.3 adBuffer.c\DTcomment{Tapenade support file}. 
.3 adBuffer.f\DTcomment{Tapenade support file}. 
.3 adBuffer.h\DTcomment{Tapenade support file}. 
.3 adStack.c\DTcomment{Tapenade support file}. 
.3 adStack.h\DTcomment{Tapenade support file}. 
.3 air\_adj.F\DTcomment{Driver for reverse-mode adjoint flow code}. 
.3 airfoil.F\DTcomment{Driver for non-linear flow code}. 
.3 air\_lin.F\DTcomment{Driver for forward-mode linear flow code}. 
.3 flux\_face\_b.f\DTcomment{ADJ version of flux\_face from routines.F}. 
.3 flux\_face\_bx.f\DTcomment{Ditto flux\_face\_b with co-ordinate variations}. 
.3 flux\_face\_d.f\DTcomment{TL version of flux\_face from routines.F}. 
.3 flux\_face\_dx.f\DTcomment{Ditto flux\_face\_d with co-ordinate variations}. 
.3 flux\_wall\_b.f\DTcomment{ADJ version of flux\_wall from routines.F}. 
.3 flux\_wall\_bx.f\DTcomment{Ditto flux\_wall\_b with co-ordinate variations}. 
.3 flux\_wall\_d.f\DTcomment{TL version of flux\_wall from routines.F}. 
.3 flux\_wall\_dx.f\DTcomment{Ditto flux\_wall\_d with co-ordinate variations}. 
.3 input.F\DTcomment{Routines to read and write data}. 
.3 lift\_wall\_b.f\DTcomment{ADJ version of lift\_wall from routines.F}. 
.3 lift\_wall\_bx.f\DTcomment{Ditto lift\_wall\_b with co-ordinate variations}. 
.3 lift\_wall\_d.f\DTcomment{TL version of lift\_wall from routines.F}. 
.3 lift\_wall\_dx.f\DTcomment{Ditto lift\_wall\_d with co-ordinate variations}. 
.3 print\_active.F\DTcomment{Routines to pretty print matrices and vectors}. 
.3 routines.F\DTcomment{Real and complex versions of non-linear routines}. 
.3 testlinadj.F\DTcomment{Derivatives test code}. 
.3 time\_cell\_b.f\DTcomment{ADJ version of time\_cell from routines.F}. 
.3 time\_cell\_bx.f\DTcomment{Ditto time\_cell\_b with co-ordinate variations}. 
.3 time\_cell\_d.f\DTcomment{TL version of time\_cell from routines.F}. 
.3 time\_cell\_dx.f\DTcomment{Ditto time\_cell\_d with co-ordinate variations}. 
.3 const.inc\DTcomment{Constants common block}. 
.3 flow.dat\DTcomment{Initial flow - scaled up problem size}. 
.3 flow.dat.bak\DTcomment{Initial flow - original}. 
.3 grid.dat\DTcomment{Unstructured grid - scaled up problem size}. 
.3 grid.dat.bak\DTcomment{Unstructured grid - original}. 
.3 Makefile\DTcomment{Build commands}. 
}
\clearpage
\subsubsection{Differentiated code using OpenAD in Reverse Joint Mode}
For details on reverse joint mode refer \cite{Griewank_2008} and \cite{Utke_2014}.\\

\noindent The directory \texttt{air\_foil\_wopenad\_joint} contains the files to compile the original \textbf{non-linear flow code} and the \textbf{reverse-mode adjoint flow code}. The binaries (\textbf{files}), corresponding to these, on building the directory are \texttt{airfoil} (\textbf{{airfoil.F}}) and \texttt{air\_adj} (\textbf{{air\_adj.F}}) respectively. \\

\begin{TodoPar}\noindent The binary \texttt{testlinadj} (\textbf{{testlinadj.F}}), which was earlier used to validate the forward-linear and reverse-adjoint routines in the \texttt{air\_foil\_tapenade} subdirectory, is not yet available to test OpenAD/F as modifications have to be done to use \texttt{oad\_active} type in the source code.\end{TodoPar}

\noindent The adjoint version of the original non-linear flow routines are used in the file  \texttt{air\_adj.F}. These have been obtained by passing the undifferentiated routines to OpenAD/F in reverse-joint mode.\\

\noindent For details on how to call OpenAD/F in reverse-joint mode refer \cite{Utke_2014}. The listing of files and their descriptions follow.\\

\dirtree{%
.1 /. 
.2 air\_foil\_wopenad\_joint.  
.3 adStack.c.bak\DTcomment{Unused Tapenade support file}. 
.3 air\_adj.F\DTcomment{Driver for reverse-mode adjoint flow code}. 
.3 airfoil.F\DTcomment{Driver for non-linear flow code}. 
.3 air\_lin.F.bak\DTcomment{Unused forward-mode linear flow code driver}. 
.3 const.inc\DTcomment{Constants common block}. 
.3 flow.dat\DTcomment{Initial flow - scaled up problem size}. 
.3 flow.dat.bak\DTcomment{Initial flow - original}. 
.3 flux\_face.F\DTcomment{UD version of flux\_face from routines.F}. 
.3 flux\_wall.F\DTcomment{UD version of flux\_wall from routines.F}. 
.3 grid.dat\DTcomment{Unstructured grid - scaled up problem size}. 
.3 grid.dat.bak\DTcomment{Unstructured grid - original}. 
.3 iaddr.c\DTcomment{OpenAD/F support file}. 
.3 input.F\DTcomment{Routines to read and write data}. 
.3 lift\_wall.F\DTcomment{UD version of lift\_wall from routines.F}. 
.3 Makefile\DTcomment{Build commands}. 
.3 Makefile.bak\DTcomment{Unused makefile}. 
.3 print\_active.F\DTcomment{Routines to pretty print matrices and vectors}. 
.3 routines.F\DTcomment{Real and complex versions of non-linear routines}. 
.3 testlinadj.F.bak\DTcomment{Unused derivatives test code}. 
.3 time\_cell.F\DTcomment{UD version of time\_cell from routines.F}.  
}

\clearpage
\subsubsection{Differentiated code using OpenAD in Reverse Split Mode}
For details on reverse split mode refer \cite{Griewank_2008} and \cite{Utke_2014}.\\

\noindent The directory \texttt{air\_foil\_wopenad\_split} contains the files to compile the original \textbf{non-linear flow code} and the \textbf{reverse-mode adjoint flow code}. The binaries (\textbf{files}), corresponding to these, on building the directory are \texttt{airfoil} (\textbf{{airfoil.F}}) and \texttt{air\_adj} (\textbf{{air\_adj.F}}) respectively. \\

\begin{TodoPar}\noindent The binary \texttt{testlinadj} (\textbf{{testlinadj.F}}), which was earlier used to validate the forward-linear and reverse-adjoint routines in the \texttt{air\_foil\_tapenade} subdirectory, is not yet available to test OpenAD/F as modifications have to be done to use \texttt{oad\_active} type in the source code.\end{TodoPar}

\noindent The adjoint version of the original non-linear flow routines are used in the file  \texttt{air\_adj.F}. These have been obtained by passing the undifferentiated routines to OpenAD/F in reverse-split mode.\\

\noindent For details on how to call OpenAD/F in reverse-split mode refer \cite{Utke_2014}. The listing of files and their descriptions follow.\\

\dirtree{%
.1 /. 
.2 air\_foil\_wopenad\_split.  
.3 adStack.c.bak\DTcomment{Unused Tapenade support file}. 
.3 air\_adj.F\DTcomment{Driver for reverse-mode adjoint flow code}. 
.3 airfoil.F\DTcomment{Driver for non-linear flow code}. 
.3 air\_lin.F.bak\DTcomment{Unused forward-mode linear flow code driver}. 
.3 const.inc\DTcomment{Constants common block}. 
.3 flow.dat\DTcomment{Initial flow - scaled up problem size}. 
.3 flow.dat.bak\DTcomment{Initial flow - original}. 
.3 flux\_face.F\DTcomment{UD version of flux\_face from routines.F}. 
.3 flux\_wall.F\DTcomment{UD version of flux\_wall from routines.F}. 
.3 grid.dat\DTcomment{Unstructured grid - scaled up problem size}. 
.3 grid.dat.bak\DTcomment{Unstructured grid - original}. 
.3 iaddr.c\DTcomment{OpenAD/F support file}. 
.3 input.F\DTcomment{Routines to read and write data}. 
.3 lift\_wall.F\DTcomment{UD version of lift\_wall from routines.F}. 
.3 Makefile\DTcomment{Build commands}. 
.3 Makefile.bak\DTcomment{Unused makefile}. 
.3 print\_active.F\DTcomment{Routines to pretty print matrices and vectors}. 
.3 routines.F\DTcomment{Real and complex versions of non-linear routines}. 
.3 testlinadj.F.bak\DTcomment{Unused derivatives test code}. 
.3 time\_cell.F\DTcomment{UD version of time\_cell from routines.F}.  
}

\clearpage
\subsubsection{Differentiated code using OpenAD in Forward Mode}
\noindent The directory \texttt{air\_foil\_wopenad\_tanglin} contains the files to compile the original \textbf{non-linear flow code} and the \textbf{forward-mode linear flow code}. The binaries (\textbf{files}), corresponding to these, on building the directory are \texttt{airfoil} (\textbf{{airfoil.F}}) and \texttt{air\_lin} (\textbf{{air\_lin.F}}) respectively. \\

\begin{TodoPar}\noindent The binary \texttt{testlinadj} (\textbf{{testlinadj.F}}), which was earlier used to validate the forward-linear and reverse-adjoint routines in the \texttt{air\_foil\_tapenade} subdirectory, is not yet available to test OpenAD/F as modifications have to be done to use \texttt{oad\_active} type in the source code.\end{TodoPar}

\noindent The tangent-linear version of the original non-linear flow routines are used in the file  \texttt{air\_lin.F}. These have been obtained by passing the undifferentiated routines to OpenAD/F in forward mode.\\

\noindent For details on how to call OpenAD/F in forward-mode refer \cite{Utke_2014}. The listing of files and their descriptions follow.\\

\dirtree{%
.1 /. 
.2 air\_foil\_wopenad\_tanglin.  
.3 airfoil.F\DTcomment{Driver for non-linear flow code}. 
.3 air\_lin.F\DTcomment{Driver for forward-mode linear flow code driver}. 
.3 const.inc\DTcomment{Constants common block}. 
.3 flow.dat\DTcomment{Initial flow - scaled up problem size}. 
.3 flow.dat.bak\DTcomment{Initial flow - original}. 
.3 flux\_face.F\DTcomment{UD version of flux\_face from routines.F}. 
.3 flux\_wall.F\DTcomment{UD version of flux\_wall from routines.F}. 
.3 grid.dat\DTcomment{Unstructured grid - scaled up problem size}. 
.3 grid.dat.bak\DTcomment{Unstructured grid - original}. 
.3 iaddr.c\DTcomment{OpenAD/F support file}. 
.3 input.F\DTcomment{Routines to read and write data}. 
.3 lift\_wall.F\DTcomment{UD version of lift\_wall from routines.F}. 
.3 Makefile\DTcomment{Build commands}. 
.3 Makefile.bak\DTcomment{Unused makefile}. 
.3 print\_active.F\DTcomment{Routines to pretty print matrices and vectors}. 
.3 routines.F\DTcomment{Real and complex versions of non-linear routines}. 
.3 testlinadj.F.bak\DTcomment{Unused derivatives test code}. 
.3 time\_cell.F\DTcomment{UD version of time\_cell from routines.F}. 
}
\clearpage
\subsubsection{Mesh generation}\label{mesh_gen}
The directory \texttt{mesh\_generator} packs a \texttt{MATLAB} script \texttt{naca0012.m} which can be used to generate larger meshes. The \texttt{airfoil} example can read the meshes that are in the ``old'' format. This can be specified while calling the script from \texttt{MATLAB} like below.
\hfill\break
\begin{lstlisting}[language=mymatlab, numbers=none]
>> naca0012(`old')
\end{lstlisting}
The script generates a \texttt{grid.dat} file where the first row contains the number of nodes, cells and edges in the mesh and the corresponding mesh data follows. The size of the mesh can be controlled by changing the parameters \texttt{I} and \texttt{J} in the \texttt{MATLAB} script.

\begin{NotePar}
\noindent To use a new grid file in airfoil code, the parameters \texttt{maxnode}, \texttt{maxcell} and \texttt{maxedge} have to be set appropriately in each of the driver files. Also, note that the \texttt{maxcell} is one more than corresponding value from the \texttt{grid.dat} file.
\end{NotePar}
\subsection{Modifications performed}
Refer section \ref{diff_airfoil}
\subsection{How to build}
Running make as below, in each of the four subdirectories beginning with ``airfoil'' will build the  binaries \texttt{airfoil}, \texttt{air\_lin} and \texttt{air\_adj}.
\hfill\break
\begin{lstlisting}[language=mybash, numbers=none]
    $ make
\end{lstlisting}
\begin{NotePar}
\noindent  The directories \texttt{air\_foil\_wopenad\_split} and \texttt{air\_foil\_wopenad\_joint} will only build the binaries \texttt{airfoil} and \texttt{air\_adj}.\\

\noindent Likewise the directory \texttt{air\_foil\_wopenad\_tanglin} will only build the binaries \texttt{airfoil} and \texttt{air\_lin}.
\end{NotePar}
\subsection{How to verify}
At the time of writing, there exists no script that can validate the output from any of the binaries. All versions of the binary \texttt{airfoil} should produce exactly the same output.\\

\noindent Validation by eyeballing the output from each version of the binaries \texttt{air\_lin} and  \texttt{air\_adj} has been performed. It is of significance to note that the outputs from \texttt{air\_lin} and  \texttt{air\_adj} should also be within certain fixed tolerance from one another.

\begin{TodoPar}
\noindent It will be valuable to write a \texttt{python} script that will take as input two \texttt{csv} files and find the \texttt{max-norm} of the difference between the corresponding entries. Other norms may also be computed. 
\end{TodoPar}

\noindent In order to test the derivatives, since the code does computations deterministically i.e. there are no convergence related iterations, each version of \texttt{airfoil} can be made to write the computed derivatives to a file and the files themselves can be passed to the \texttt{python} script.\\

\begin{TodoPar}
\noindent Additionally, the binary \texttt{testlinadj} can be used to test linear and adjoint routines in \texttt{air\_foil\_tapenade} directory. And by modifying \texttt{testlinadj.F} suitably - by making use of \texttt{oad\_active} type - in each of the other directories, validation can be performed.
\end{TodoPar}
\subsection{How to extend}
\subsubsection{Generating larger grids and flows}
Section \ref{mesh_gen} discusses how to create a larger mesh. Once a larger mesh has been created, corresponding flow has to be now generated. The \texttt{airfoil} application provides all the necessary code to do this. \\

\noindent The file \texttt{input.F} has a section of code that goes as below.\\

\begin{lstlisting}[language=myfortran]
c
c------ read in data from flow file, initialising if necessary
c
        open(1,file='flow.dat',status='old')
        read(1,*) p, r, mach, alpha
        alpha = alpha*datan(1.0d0)/45.0d0
        p = 1.0d0
        r = 1.0d0
        u = dsqrt(gam*p/r)*mach
        e = p/(r*gm1) + 0.5d0*u**2
c
        do ic = 1, ncell
          q(1,ic) = r
          q(2,ic) = r*u
          q(3,ic) = 0.d0
          q(4,ic) = r*e
        enddo
c
        do ic = 1, ncell
          read(1,*,err=999,end=999) (q(ipde,ic),ipde=1,4)
        enddo
 999    close(1)
\end{lstlisting}
\hfill \break
In order to generate the corresponding flows, lines \texttt{19-22} of \texttt{input.F} may be initially commented. A base flow is initialized in lines \texttt{12-17}. Using this, the binary \texttt{airfoil} may be executed for a long time period. When the binary finishes executing, it writes the solution at the end of the simulation to \texttt{flow.dat} file. This file may later be used by uncommenting the lines \texttt{19-22} of \texttt{input.F} to warm-start the simulation with a hopefully converged flow.\\

\noindent Paraphrasing an email from one of the authors of \cite{Giles_2005}:

\begin{Verbatim}[xleftmargin=2em]
 The airfoil testcase was developed several years ago for an
 an AD project using Tapenade:
 http://people.maths.ox.ac.uk/~gilesm/codes/airfoil/index.html

 The input.F file initializes a uniform flow at a given 
 pressure, density, mach number and angle of attack. For 
 this technique (of initializing the flow) to be used with
 larger meshes, the timestep in the CFD calculations may
 need to be  lowered to handle strong initial transients.

 The authors used to run the simulation using just this
 technique, letting the simulation warm up by running 
 several iterations. At the end of each simulation the
 updated flow is written to the same flow.dat file which
 can later be used to continue the simulation, if needed.
\end{Verbatim}
\subsubsection{Throwing more stuff on the tape, artificially}
One of the intended uses of this airfoil test-case is to test checkpointing algorithms in AD tools. With regards to \texttt{OpenAD/F}